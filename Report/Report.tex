\documentclass[12pt,a4paper]{report}

\usepackage[utf8]{inputenc}
\usepackage{geometry}
\geometry{a4paper, margin=1in}
\usepackage{graphicx}
\usepackage{amsmath}
\usepackage{hyperref}
\usepackage{setspace}
\usepackage{natbib}

\title{Actuarial Climate Index for the UK}
\author{Kgosi Ruri Molebatsi}
%\institute{Heriot-Watt University, Edinburgh}
\date{\today}

\begin{document}

\maketitle

\begin{abstract}
This report presents the development and analysis of an Actuarial Climate Index (ACI) tailored for the United Kingdom. 
Using climate data sourced from reputable meteorological agencies, we construct a composite index that captures the frequency and severity of extreme climate events relevant to actuarial risk assessment. 
The methodology involves data preprocessing, statistical analysis, and index construction, as detailed in the accompanying Jupyter notebook \texttt{final.ipynb}. 
Furthermore, the integration of the ACI into risk loading calculations is explored, demonstrating how climate-related risks can be quantitatively incorporated into insurance premium setting. 
Key findings indicate notable trends in temperature extremes, precipitation, and wind events over recent decades, with implications for insurance and risk management sectors. 
The report discusses the limitations of the current approach and suggests directions for future research to enhance the robustness and applicability of the UK ACI.
\end{abstract}

\tableofcontents
\listoffigures
\listoftables

\chapter{Introduction}
\section{Background}
\section{Objectives}
\section{Structure of the Report}

\chapter{Literature Review}
\section{Relevant Studies}
Several studies have explored the development and application of climate indices for risk assessment. \citet{brooks2014actuarial} introduced the Actuarial Climate Index (ACI) for North America, providing a framework for quantifying climate-related risks relevant to the insurance industry. Their methodology combines multiple climate variables, such as temperature extremes, precipitation, and wind speed, to create a composite index.

\citet{zwiers2011climate} examined the relationship between climate variability and insurance losses, highlighting the importance of robust indices for actuarial applications. Similarly, \citet{alexander2006global} conducted a global analysis of climate extremes, offering insights into trends and variability that inform index construction.

In the UK context, \citet{kendon2014uk} analyzed changes in UK rainfall and temperature extremes, providing a foundation for region-specific climate indices. The work of \citet{palin2016future} further assessed future projections of extreme weather events in the UK, emphasizing the need for dynamic risk assessment tools.

These studies collectively inform the methodology and relevance of constructing an Actuarial Climate Index tailored for the UK.
\section{Gaps in the Literature}

\chapter{Methodology}
\section{Research Design}
\section{Data Collection}
\section{Data Analysis}

\chapter{Results}
\section{Findings}
\section{Interpretation}

\chapter{Discussion}
\section{Implications}
\section{Limitations}

\chapter{Conclusion}
\section{Summary}
\section{Recommendations}
\section{Future Work}

\bibliographystyle{plainnat}
\bibliography{references}

\appendix
\chapter{Appendix}
Additional material, data, or code.

\end{document}